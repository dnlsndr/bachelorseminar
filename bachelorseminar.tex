\documentclass[english,runningheads, letterpaper]{llncs}[2018/03/10]
\usepackage{fontspec}
\setmonofont[Scale=0.8]{Dank Mono}
\usepackage{fancyhdr}
\pagestyle{fancy}
\usepackage[unicode=true]
 {hyperref}

\usepackage{listings}
\usepackage{color}
\usepackage{textcomp}
\makeatletter

\lstdefinelanguage[ECMAScript2015]{JavaScript}[]{JavaScript}{
  morekeywords=[1]{await, async, case, catch, class, const, default, do,
    enum, export, extends, finally, from, implements, import, instanceof,
    let, static, super, switch, throw, try},
  morestring=[b]` % Interpolation strings.
}

\lstdefinelanguage{JavaScript}{
  morekeywords=[1]{break, continue, delete, else, for, function, interface, export, namespace, type, if, in,
    new, return, this, typeof, var, let, const, void, while, with},
  % Literals, primitive types, and reference types.
  morekeywords=[2]{false, null, true, boolean, number, string, undefined,
    Array, Boolean, Date, Math, Number, String, Object},
  % Built-ins.
  morekeywords=[3]{eval, parseInt, parseFloat, escape, unescape},
  sensitive,
  morecomment=[s]{/*}{*/},
  morecomment=[l]//,
  morecomment=[s]{/**}{*/}, % JavaDoc style comments
  morestring=[b]',
  morestring=[b]"
}[keywords, comments, strings]


\lstalias[]{ES6}[ECMAScript2015]{JavaScript}

% Requires package: color.
\definecolor{mediumgray}{rgb}{0.5, 0.5, 0.5}
\definecolor{mediumblue}{rgb}{0.0, 0.0, 0.8}
\definecolor{forestgreen}{rgb}{0.13, 0.55, 0.13}
\definecolor{darkviolet}{rgb}{0.58, 0.0, 0.83}
\definecolor{royalblue}{rgb}{0.25, 0.41, 0.88}
\definecolor{crimson}{rgb}{0.86, 0.8, 0.24}

\lstdefinestyle{JSES6Base}{
  backgroundcolor=\color{white},
  basicstyle=\ttfamily,
  breakatwhitespace=true,
  breaklines=true,
  captionpos=b,
  columns=fullflexible,
  commentstyle=\color{mediumgray}\upshape,
  emph={},
  emphstyle=\color{crimson},
  extendedchars=true,  % requires inputenc
  fontadjust=true,
  frame=single,
  framesep=6pt, 
  identifierstyle=\color{black},
  keepspaces=true,
  keywordstyle=\color{mediumblue},
  keywordstyle={[2]\color{darkviolet}},
  keywordstyle={[3]\color{royalblue}},
  numbers=left,
  numbersep=10pt,
  numberstyle=\tiny\color{mediumgray},
  rulecolor=\color{mediumgray},
  showlines=true,
  showspaces=false,
  showstringspaces=false,
  showtabs=false,
  stringstyle=\color{forestgreen},
  tabsize=2,
  title=\lstname,
  upquote=true  % requires textcomp
}

\lstdefinestyle{JavaScript}{
  language=JavaScript,
  style=JSES6Base
}
\lstdefinestyle{ES6}{
  language=ES6,
  style=JSES6Base
}


\raggedbottom

\begin{document}

\counterwithin{lstlisting}{subsection}
\counterwithin{note}{subsection}
\counterwithin{question}{subsection}
\counterwithin{solution}{subsection}
\counterwithin{problem}{subsection}
\counterwithin{remark}{subsection}

\title{Typescript: Gradually typing JavaScript}
\author{Daniel~Schneider}
\authorrunning{D. Schneider}
\institute{Ludwig-Maximilians-University, Geschwister-Scholl-Platz 1, 80539 Munich, Germany \email{poststelle@verwaltung.uni-muenchen.de}}
\maketitle
\begin{abstract}
``TypeScript is an open-source language which builds on JavaScript, one of the world’s most used tools, by adding static type definitions.''\cite{ref_ts_homepage}

This paper will discuss the benefits TypeScript can give developers
in terms of software quality and security. It will highlight examples
of the main features that are part of the TypeScript standard and
how those can be leveraged to convert a previous JavaScript program
into a safer TypeScript program.

\keywords{TypeScript \and JavaScript \and Type safety \and Gradual typing} 
\end{abstract}

\section{Introduction}

Programming is the act or process of planning or writing a program \cite{ref_programing_def}.
The general goal of programming languages is to make it easier for
humans to interact with computers in a comprehensible way \cite{ref_def_programing_language}. This results
in programming languages that increasingly abstract away complex mechanisms
to speed up or simplify program development. A relatively recent phenomena
is the complete removal of type annotations within programming languages,
whereby the compiler completely infers variable types on a best effort principle. This gives
the developer a simpler work-flow by leaving out the mental step of
thinking about typings. Although this may first seem to be a good
thing, it has become increasingly apparent, that bigger programs degrade
in software quality if software quality is not a priority. This can
be mitigated by using typing systems with static type checking, but
would require a complete rewrite in a type annotated and type checked
language. One of the most popular scripting languages out there --
JavaScript -- does away with type annotations, which to the demise
of big software projects that are written in JavaScript, results in degrading
software quality quite quickly if it is not taken care of. \\
TypeScript was introduced to remedy this fundamental issue of JavaScript.
It allows for code type annotations and static type checking of the
program, while still being able to infer untyped code
areas to the point of compiling plain JavaScript code without any
type annotations. 

\clearpage

To get a better feeling for what benefits TypeScript can give us,
we will have a look at a possibly problematic JavaScript snippet:

\begin{lstlisting}[style=ES6, language=JavaScript, caption="Problematic JavaScript code snippet"]
let anotherNumber = "13"
let numberList = [1,2,3,5,8]

function insertNumberAndReduce(list, newNumber){
  list.push(newNumber)
  return list.reduce((acc, val) => acc + val)
}

let result = insertNumberAndReduce(numberList, anotherNumber)
\end{lstlisting}

When executing this code snippet, we get the result ``1913''
which obviously is false. To better understand what is happening here,
we will follow this code in a more abstract manner:

\begin{enumerate}
\item String ``13'' gets inserted into the number list, consisting of
integers
\item ``insertNumberAndReduce'' reduces all numbers by adding them
\item ``13'' is added on to the sum of all preceding numbers in the numberList
\item Adding strings and numbers is not possible, therefore the sum of the
numbers is cast into a string and both strings are concatenated
\item We receive the result ``1913''
\end{enumerate}

Since in JavaScript there is no system that can check at compile time if what we are doing may or may not
be intended, it tries its best to compute our code, potentially casting variables into different data-types. Of course we could manually check if the parameters given to the function correspond to native JavaScript data-types by using conditionals, but this would  degrade code legibility, may impact run-time performance negatively and might be omitted by developers therefore also requiring thorough testing.

\begin{claim}
This issue could easily be prevented by keeping the developer from inserting
a string into a list of numbers.
\end{claim}

\pagebreak

Let us now rewrite the example into TypeScript with the corresponding
type annotations and see if we are able to compile this snippet successfully:

\begin{lstlisting}[style=ES6, language=JavaScript, caption="Fixed TypeScript code snippet"]
let anotherNumber = "13"
let numberList = [1,2,3,5,8]

function insertNumberAndReduce
(list: number[], newNumber: number): number {
  list.push(newNumber)
  return list.reduce((acc, val) => acc + val)
}

let result = insertNumberAndReduce(numberList, anotherNumber)
\end{lstlisting}

\begin{note}
We have added type annotations to the function definition such that
the function will only accept a list of numbers, a additional number
and will only return a value of type number
\end{note}

Executing compilation, TypeScript will throw an error, since the function
call does not comply with the function definition and its
type annotations.
This example clearly highlights the strengths of TypeScript in an otherwise untyped JavaScript landscape.


\section{Typescript in a nutshell}

Before we have a look at all the features TypeScript provides to us, we should first get to know it better. This will help understanding what TypeScript does in the background and how it should be handled.

\pagebreak
\subsection{History \cite{ref_ts_wikipedia}}

\begin{multicols}{2}
[
\noindent
To get to know TypeScript a bit better, we will have a quick look at its history.
]
\noindent
Microsoft initially created Typescript v0.8 in 2012 and published it to the public. 

\medskip
\noindent
In the following year they released v0.9 which brought with it a generics system for type annotations. 

\medskip
\noindent
In 2014 Microsoft then released the first production-ready version v1.0 integrating the TypeScript compiler into their already existing Visual Sudio IDE platform, bringing TypeScript to a broader audience of developers that might want to try TypeScript. 

\medskip
\noindent
Later that year, in July, a new compiler version was released with five times the performance of the old compiler, improving production implementaion. 

\medskip
\noindent
In 2016 v2.0 was released, that tried to remedy a general programing language issue commonly called the "Billion-Dollar mistake"\footnote{The term "Billion-Dollar mistake" was first coined by Tony Hoare, who also created the ALGOL W language. Having implemented null referencing in that language he may have caused around a billion dollars of damage since 1965, due to errors, vulnerabilities and system crashes. In the JavaScript world the implementation of the primitive "null" datatype may have also caused high sums of damages due to a around 25 year old JavaScript bug that causes the primitive "null" datatype to resolve to the type of "object"\cite{ref_billion_dollar_mistake}}. 

\medskip
\noindent
Later v3.0 was released in 2018 brining with it several smaller features like tuples. 

\medskip
\noindent
The most recent version v4.0 was released 2020 and brings improvements to the existing Tuple system by adding Variadic Tuples.
\end{multicols}

\subsection{Gradual typing \cite{ref_gradual_typing}}

TypeScript leverages gradual typing. Gradual typing is the principle
of gradually adapting type annotations in your program, with the ability
to even completely leave out type annotations. This enables developers to
gradually migrate over JavaScript projects to TypeScript without having
to worry about code compatibility. TypeScript will try to infer as
many types as possible, but \textit{cannot guarantee} complete type safety
as long as there are sections in a program that are not type-annotated. 
This enables us to selectively type certain areas of the code statically \footnote{Static typing is the principle of annotating and defining types within your code and letting a compiler check these types for soundness in a separate compilation step \ref{static_type_checking}\cite{ref_def_static_dynamic_type_checking}}, while typing other areas of the code dynamically \footnote{Dynamic type checking is the process of checking the types witin a program at runtime, potentially throwing an exception if a type error is encountered \cite{ref_def_static_dynamic_type_checking}}.

\subsection{Static type checking \label{static_type_checking}}

During compilation of the TypeScript source code, the Type-resolver \ref{architecture} checks for type soundness by using Static type checking algorithms.
Static type checking is the process of analyzing code -- by using a previously generated AST\footnote{``An abstract syntax tree (AST) is a way of representing the syntax of a programming language as a hierarchical tree-like structure. This structure is used for generating symbol tables for compilers and later code generation. The tree represents all of the constructs in the language and their subsequent rules.'' \cite{ref_def_ast}} -- for soundness at compile time. Should the Type-resolver encounter any type soundness issues, it will throw a compilation exception and will halt any further computations until the issues are fixed and the compiler is restarted. In the case of TypeScript, the Type-resolver has be able to cope with missing type annotations within the code and must also be able to infer types of variables if they are not type-annotated to guarantee gradual typing capabilities. \cite{ref_static_type_checking}

\subsection{Architectural overview \label{architecture}}

TypeScript itself is primarily a compiler that runs on preexisting TypeScript code. This compiler takes in TypeScript files and emits valid JavaScript, which can then be executed by JavaScript run-times such as Node.js or Deno.

\medskip
\noindent
\begin{list}{•}{These are the building blocks of the TypeScript compiler\cite{ref_ts_architecture}:}

\item \textbf{Pre-processor:} The Pre-processor creates the compilation context by evaluating and resolving all ``.ts'' dependency references and potentially importing type files (.d.ts) 

\item \textbf{Parser:} The parser parses the TypeScript code syntax into a Abstract Syntax Tree short AST.

\item \textbf{Binder:} The Binders responsibility is to connect the various parts of the code into a coherent type system by creating Symbols.

\item \textbf{Type-resolver/-checker:} The Type-resolver checks the code for type soundness and throws an exception for any type collisions, resulting in a failed compilation. This is the central part of the typescript compiler enabling static type checking.

\item \textbf{Emitter:} The emitter then generates native JavaScript code from the previously generated AST, which can be executed by any capable JavaScript runtime.

\end{list}

\begin{note}
Compilation by compiler is done in exactly the same order as the architectural building blocks are listed here.
\end{note}

\begin{note}
There are also other parts in the TypeScript architecture, which are out of the scope of this paper. If you are interested in reading up more about the those other parts, please follow the URL in reference \cite{ref_ts_architecture}
\end{note}

\section{Features}

Let us now have a look at the four most important features TypeScript
provides.

\subsection{Interfaces and Types}

Interfaces and Types are the most basic building blocks within TypeScript
and allow us to define custom object and type schemas. First we will
have a look at how Interfaces and Types are defined and referenced
in code:

\begin{lstlisting}[style=ES6, language=JavaScript, caption="Interface and Type declarations"]
interface User {
	name: string
	email: string
	age?: number
}
// An equivalent to the interface can also be defined as a type
type User = {
	name: string
	email: string
	age?: number
}
\end{lstlisting}

\begin{question}
What is the difference between an Interface and a Type?
\end{question}

\begin{solution}
There are hardly any differences between Interfaces and Types. Types
have the ability to define \textit{primitive} types such as numbers and strings,
while Interfaces can only define object structures and types within.
Types can not be merged with each other, while Interfaces have the
capability to be merged, should the developer define several Interfaces
with the same name \cite{ref_ts_interfaces_vs_types}. In general, Interfaces should be preferred to
Types if they can be used interchangeably.
\end{solution}

These Interfaces and Types can then be used in the following manner:

\begin{lstlisting}[style=ES6, language=JavaScript, caption="Interface and Type usage"]
const user: User = {
	name: "John Doe",
	email: "john@doe.com"
}
\end{lstlisting}

\begin{note}
The ``age'' key can be omitted, since we've made the key optional
in the type definition by suffixing the key with a question-mark
\end{note}


\subsection{Tuples}

Tules give you the ability to type the structure of an array. This
means that it is possible to define a specific type for each index
of that array and enabling us to build tuples out of several other
tuple building blocks, called variadic tuples. In the following example
we can see how simple Tuples can be implemented:

\begin{lstlisting}[style=ES6, language=JavaScript, caption="Tuple usage"]
// Type structure: <order>, <key>, <value>
type Detail = [number, string, number | string]
let userDetails: Detail[] = [
	[1, 'age', 22],
	[2, 'semester', 6],
	[10, 'name', 'John Doe']
]
\end{lstlisting}

\begin{note}
Using a pipe symbol (|) between types resembles an OR operator. Therefore
the type of that element can be either a number OR a string
\end{note}

By using Tuples, TypeScript can prevent wrong usage of arrays:

\begin{lstlisting}[style=ES6, language=JavaScript, caption="Problematic Tuple usage"]
let userDetails: Detail[] = [
	[1, 'age', 22, 'a fourth element'],
	[2, 5, 6],
	['some order', 'name', 'John Doe']
]
\end{lstlisting}

\begin{problem}
TypeScript will throw an error upon compilation, since the Arrays
within the 'userDetails' Array do not correspond to the 'Detail' type
\end{problem}

Finally we also have the ability to create Tuples out of other Tuples.
This is handy if one e.g. knows the first elements of an array but
wants to dynamically define all following element types of that Array.
These Variadic Tuples can reduce code complexity. Rewriting our previous
working example into Variadic Tuples results in the following code:

\begin{lstlisting}[style=ES6, language=JavaScript, caption="Variadic Tuples"]
// Type structure: <order>, <key>, <custom element types>
type Detail<T extends unknown[]> = [number, string, ...T]

let userDetails: Detail<[number | string]>[] = [
	[1, 'age', 22],
	[2, 'semester', 6],
	[10, 'name', 'John Doe']
]
\end{lstlisting}

\begin{remark}
This Variadic Tuple system now lets us define the ``<custom element
types>'' at a variable basis, giving us more freedom in reusing Types,
while still forcing us to abide to general typing structures like
the 'Detail' type.
\end{remark}


\subsection{Namespaces}

Namespaces allow us to split Types into separate logical units. This
also allows for equally named types without them interfering with
each other.

\begin{lstlisting}[style=ES6, language=JavaScript, caption="Namespace definition"]
namespace User {
	export interface Profile {
		email: string
	}
}

namespace Admin {
	export interface Profile {
		email: string
		roles: Role[]
	}
	interface Role {
		name: string
	}
}
\end{lstlisting}

\begin{note}
The User and Admin namespace both have a Profile type. Although there
are now two interfaces with the same name, they do not collide with
each other since they exist in a different scope.
\end{note}

\begin{note}
Namespaces allow for private types that are only accessible within
the namespace itself. Here, the Role type within the Admin namespaces
is not exported, thus only types within the Admin namespace can reference
it. This allows for better control over the typing API of your program
by only making types visible that are intended to be used
\end{note}

To use a type within a namespace we access it through dot notation:

\begin{lstlisting}[style=ES6, language=JavaScript, caption="Namespace usage"]
const userProfile: User.Profile = {
	name: "John Doe"
}
\end{lstlisting}

\subsection{Generics}

Generics introduce an abstration layer to your typings. It allows
for dynamic types depending on the use case and rather acts as a type placeholder than a real type. Let's have a look at
how Generics are used:

\begin{lstlisting}[style=ES6, language=JavaScript, caption="Generic definition"]
function listify<T>(x: T): T[] {
	return [x]
}
\end{lstlisting}

This listify function can take any type of parameter x and returns a list of the type of parameter x. This allows for fully dynamic typing without having to reimplement the same function for each type of parameter. It is also possible to manually control the type we pass into the function. Let's have a look at the following use-case:

\begin{lstlisting}[style=ES6, language=JavaScript, caption="Generic usage"]
listify<bool>(true)
\end{lstlisting}

Here we are telling the listify function, that it should create a list of booleans, passing in one element of value "true"


\section{Conclusion}

Besides some configuration overhead and learning how to use TypeScript effectively, TypeScript can majorly improve software quality if used in the right way. Due to its gradual typing approach, preexisting JavaScript code need not be touched for one to get started with TypeScript. It can find bugs in your code and give you direct feedback while writing it, leveraging the power of the compiler. These bugs would normally have to be tested against in elaborate testing setups that may even take up more development time than simply implementing TypeScript in error prone program areas. Given all the features TypeScript has, there are hardly any limits to type definitions. All in all it can be said that TypeScript is a great tool to keep up and prevent degradation of the quality of large software projects if one is willing to take on the burden of additionally maintaining types in the code-base.  

\pagebreak

\begin{thebibliography}{1}
\bibitem{ref_billion_dollar_mistake} Medium: Billion-dollar mistake(s)!, \href{https://jagadeeshrampam.medium.com/billion-dollar-mistake-s-37620be56a12}{https://jagadeeshrampam.medium.com/billion-dollar-mistake-s-37620be56a12}. Last accessed August 11, 2021

\bibitem{ref_def_programing_language} Definition: Function and target, \href{https://en.wikipedia.org/wiki/Programming\_language}{https://en.wikipedia.org/wiki/Programming\_language}. Last accessed August 11, 2021

\bibitem{ref_ts_wikipedia} Typescript Wikipedia page, \href{https://en.wikipedia.org/wiki/TypeScript}{https://en.wikipedia.org/wiki/TypeScript}. Last accessed August 13, 2021

\bibitem{ref_def_ast} Abstract Syntax Tree (AST), \href{https://www.techopedia.com/definition/22431/abstract-syntax-tree-ast}{https://www.techopedia.com/definition/22431/abstract-syntax-tree-ast}. Last accessed August 11, 2021

\bibitem{ref_ts_homepage} Typescript Homepage, \href{https://www.typescriptlang.org/}{https://www.typescriptlang.org/}. Last accessed August 8, 2021

\bibitem{ref_gradual_typing} Gradual Typing: A New Perspective, \href{https://dl.acm.org/doi/pdf/10.1145/3290329}{https://dl.acm.org/doi/pdf/10.1145/3290329}. Last accessed August 13, 2021

\bibitem{ref_ts_architecture} Architectural Overview, \href{https://github.com/microsoft/TypeScript/wiki/Architectural-Overview}{https://github.com/microsoft/TypeScript/wiki/Architectural-Overview}. Last accessed August 8, 2021

\bibitem{ref_static_type_checking} Type Checking, \href{https://www.brainkart.com/article/Type-Checking_8086/}{https://www.brainkart.com/article/Type-Checking\_8086/}. Last accessed August 12, 2021

\bibitem{ref_ts_interfaces_vs_types} Types vs. interfaces in TypeScript, \href{https://blog.logrocket.com/types-vs-interfaces-in-typescript/}{https://blog.logrocket.com/types-vs-interfaces-in-typescript/}. Last accessed August 12, 2021

\bibitem{ref_def_static_dynamic_type_checking} Type chekcing -- Static of Dynamic?, \href{https://medium.com/@himankbh/type-checking-static-or-dynamic-76067dbccc7b}{https://medium.com/@himankbh/type-checking-static-or-dynamic-76067dbccc7b}. Last accessed August 12, 2021

\bibitem{ref_programing_def} Thesaurus programing definition, \href{https://www.dictionary.com/browse/programing}{https://www.dictionary.com/browse/programing}. Last accessed August 9, 2021
\end{thebibliography}

\end{document}
